% This is samplepaper.tex, a sample chapter demonstrating the
% LLNCS macro package for Springer Computer Science proceedings;
% Version 2.20 of 2017/10/04
%
\documentclass[runningheads]{llncs}
%
\usepackage{graphicx}
% Used for displaying a sample figure. If possible, figure files should
% be included in EPS format.
%
% If you use the hyperref package, please uncomment the following line
% to display URLs in blue roman font according to Springer's eBook style:
% \renewcommand\UrlFont{\color{blue}\rmfamily}

\begin{document}
%
\title{Técnicas y Herramientas Modernas}

\subtitle {Traducción}
%
%\titlerunning{Abbreviated paper title}
% If the paper title is too long for the running head, you can set
% an abbreviated paper title here
%
\author{Román Fernandez\inst{1} \and
Zacarías Sansone\inst{2}}
%
\authorrunning{Román Fernández y Zacarías Sansone}
% First names are abbreviated in the running head.
% If there are more than two authors, 'et al.' is used.
%
\institute{Universidad Nacional de Cuyo, Facultad de Ingeniería}

%
\maketitle              % typeset the header of the contribution
%
\begin{abstract}
La finalidad del siguiente escrito es nombrar y explicar las fuentes de energía renovables que se utilizan en la actualidad en algunos sectores del planeta. 

El objetivo es poder informar al lector sobre estas energías alternativas y limpias, para que este pueda sacar sus propias conclusiones respecto a la transición energética y al nuevo futuro que nos espera. Futuro en el cual, junto con la cooperación de la gente y de sus gobernantes, conseguiremos una reducción en las emisiones de carbono a la atmósfera y de la huella de carbono.

\keywords{Energía renovable \and Energía nuclear \and Energía Solar \and Energía Eólica \and Energía hidroeléctrica \and Bioenergía \and Geotermia \and Energía de las olas \and Energía mareomotriz.}
\end{abstract}
%
%
%
\section{Capitulo 5: La Energía y el Medio Ambiente 2: Energía nuclear y renovables}
\subsection{Uranio y Energía Nuclear}

En la actualidad existe una discrepancia en la inversión de proyectos de energía nuclear, ya que son muy costosos pero producen grandes cantidades de energía baja en carbono.

Una gran cantidad de centrales nucleares en Estados Unidos están a punto de cumplir su vida útil y muchas ya la han cumplido. Se han implementado políticas para beneficiar a estas formas de energía bajas en carbono para que puedan ser mas competitivas.

Si se concreta la sustitución de estas centrales por centrales de combustión de combustibles fósiles aumentaría enormemente las emisiones de gases a la atmósfera.

Aunque además de su elevado costo, otro gran impedimento de la producción energética en centrales nucleares es los grandes efectos negativos que han traído situaciones como la de Fukushima en Japón. Que pusieron en debate el uso de este tipo de energía, y llevaron a que muchos paises fueran desminuyendo poco a poco la utilización de las mismas. Además produjeron un rechazo muy grande de parte de la sociedad.

La energía nuclear es el resultado de la división del uranio 235, isótopo que se encuentra en las rocas. En el nucleo de la central se desintegra radiactivamente el uranio, donde desprende calor que produce vapor que hace girar una turbina y un generador. El nucleo se trata de mantener frío circulando agua en la torre.

Pero no es solo el problema la central nuclear, sino que la extracción de uranio debe ser sumamente cuidadosa para que no produzca efectos negativos en la zona.Se debe prestar suma atención también al desecho de residuos radiactivos, algunos paises no cuentan con un lugar de procesamiento especializado de los mismos. Pero el uranio posee en una libra, la misma energía que tres millones de libras de carbón. 

Entonces todo esto se pone en debate, las situaciones ya nombradas, pero ademas se debe tener en cuenta las cuestiones económicas que trae, el empleo a una gran cantidad de personas, así como que hacer con las centrales viejas.


\subsection{La Energía Eólica}

La energía del viento se recoge primero como energía mecánica y luego se convierte en energía eléctrica.

En los últimos años se ha producido un aumento espectacular de la energia eolica para generar electricidad. Gran parte del crecimiento mundial de las energias renovables en las dos ultimas decadas procede de los parques eolicos. 

Los parques eolicos estan empezando a integrar el almacenamiento en baterias, lo que da a la fuente de energia intermitente cierta capacidad de ser despachable. Hay mas energia en alta mr que en tierra porque hay menos obstaculos para interrumpir los patronesde vientos. Que produce una atrraccion de los inversores en la energia eolica marina, donde hay mayor y mas estable potencia en el viento, por ende mayor produccion energetica.

La energia eolica terrestre es de mas facil acceso y por ende mas barata.

Hay 3 tipos de sistemas de flotacion para los parques eolicos marinos, boyas, el semisumergible y la plataforma con patas de tension.

La boya esta estabilizada por un material pesado que se extiende a gran profundidad y evita la accion ddañina de las olas en la superficie.

La turbina semisumergible es para aguas poco profundas, hasta 50 metros. Se basa en la flotabilidad y el area del plano de agua para mantener la estabilidad estatica, pero lo suficientemente estable para no tener lineas de amarre y con un calado poco profundo que permite remolcar la estructura totalmente montada.
Por lo que el montaje y mantenimiento se realiza en los muelles.

La plataforma utiliza lineas de amarre tensadas, puede ser dificil de desplegar pero es muy estable una vez instalada. 

La construccion de parques eolicos en alta mar requiere muchos kilometros de obras subterraneas para transportar cables, lo que provoca pertubaciones en los ecosistemas marinos.

\subsection{Energía Solar}

El sol es la fuente de energia predominante que alimenta las actividades biofisicas de la Tierra. La energía entrante del sol es absorbida, reflejada o refractada en la atmosfera, calentando el aire, el agua, etc. En una hora llega a la Tierra suficiente energia solar para alimentar a la humanidad 1 año.
La humanidad siempre ha dependido del Sol ya que es quien produce la fotosíntesis en las plantas y eso permite el cultivo.

Los combustibles fosiles tambien son energia solar captada y almacenada a traves de la fotosintesis y transformada mediante un lento proceso quimico que puede durar hasta cientos de millones de años.

Para recoger la energia solar hay que fabricar dispositivos, la produccion de celulas fotovoltaicas requiere materias primas extraidas de las minas, producidas en plantas quimicas y semiconductores.

La energia solar tambien se recoge para calefaccionar espacios y para calentar agua. Que esto puede traer un reemplazo en la utilizacion de otro tipo de combustibles para calefaccionar, por ejemplo, disminuyendo la produccion de gases de efecto invernadero.

La industria fotovoltaica ha crecido mucho mundialmente en los ultimos años. Se ha propuesto de recolectar energía solar a travez de satélites que la transmitan a la Tierra.

Un modulo fotovoltaico tipico suele ser capazde suministrar entre 200 y 400 vatios. La potencia de salida tambien se utiliza para describir la capacidad de una instalacion de fabricacion.

Las celulas fotovoltaica son las unicas que producen energía electrica por efecto fotovoltaico, en que los fotones entrantes empujan a los electrones a traves de una tension.

El efecto fotovoltaico se profuce cuando una capa semiconductora con electrones adicionales absorbe un foton entrante. Este foton entrante aumenta la energia de los electrones en los atomos que aborben el foton y eleva la energia de los electrones de la ultima capa de valencia. Con los electrones librementes disponibles, un circuito atrae los electrones libres hacia una carga electrica o a traves de la tension.. Los contactos metalicos conductores recogen y tranportan la corriente electrica a traves del modulo fotovoltaico hasta una caja de conexiones interconectada a un sistema electrico.

Ya que se compone de un semiconductor tiene una juntura "pn" que significa positivo-negativo. La union pn se profuce cuando un semiconductor con carga positiva se interpone con otro cargado negativamente. Esta juntura es necesaria para empujar el desequilibrio de cambios y permitir que los electrones fluyan de un lugar a otro.

Estos modulos fotovoltaicos generalmente estan compuestos de silicio cristalinos ocupan un 90 por ciento de la cuota del mercado mundial.

Algunas tecnologias fotovoltaicas se venden en  mercados especializados de telecomunicaciones, aplicaciones militares y satelites. Son mejores en zonas remotas sin infraestructura electrica o donde los generadores no son competitivos o requieren demasiado mantenimiento. 

El silicio monocristalino se forma mediante el metodo Czochralski, en el que las temperaturas mas altas y las velocidades de enfriamiento mas lentas dan lugar a un proceso de formacion de oblias de silicio. Las temperaturas mas elevadas y las velocidades de enfriamiento mas lentas dan como resultado lingotes de silicio cristalino extremadamente puro.

Los lingotes se suelen dopar para darles una carga intrinseca, que puede ser tipo p o n.
A continuacion los lingotes se cortan en ladrillos de silicio y luego en obleas individuales, utilizando sierras de hilo de diamante sumergidas en una mezcla de lodo. Las obleas se graban y se limpian para eliminar los daños de la cierra.

\subsection{Energía hidroeléctrica.}

La energía hidráulica se utilizó por primera vez en China para triturar granos y romper minerales. Luego, este tipo de energía fue utilizada para generar energía eléctrica, siendo al día de hoy China, Canadá y Brasil sus principales exponentes.

Esta energía se origina por la caída del agua, donde la tarea de la central hidroeléctrica es transferir energía potencial a energía cinética. El proceso consiste en que el agua que sale del embalse fluye hacia una turbina haciéndola girar, lo cual  genera energía cinética y esta se convierte en electricidad. El movimiento activará el generador, produciendo electricidad para la red.

Hay que tener en cuenta que este tipo de energía también tiene un gran impacto en los hábitats, afectando a la flora y a la fauna. Por esta razón, algunos países, como India y Brasil, están realizando transiciones energéticas, donde las presas se promueven como fuentes de energía limpia mientras transforman los ecosistemas y los medios de vida. Las presas crean embalses que funcionan como almacenamiento de agua, control de inundaciones y uso recreativo, además que poseen baja emisiones de carbono.

También existen nuevas tecnologías que pueden proporcionar energía mediante la instalación de turbinas in-pipe cerca de las instalaciones de agua. Esta tecnología produce energía renovable y, al mismo tiempo, no tiene un impacto permanente en el lugar de la instalación. Las turbinas esféricas pueden instalarse dentro de tuberías de agua alimentadas por gravedad.

\subsection{Bioenergía: Biocombustibles, biomasa, biogás y biocarbón.
}

Los procesos biológicos son muy útiles para la descarbonización, ya que las especies vegetales almacenan la energía del sol a través de la fotosíntesis. Este tipo de energía almacenada alcanza valores muy altos, y si no se la utiliza acaba irradiandose en forma de calor y se liberará carbono en los materiales orgánicos.


Los biocombustibles son combustibles que pueden ser sólidos, líquidos o gaseosos, y son producidos a partir de materiales orgánicos basados en el carbono. Los cultivos energéticos son especies de cultivos de alto rendimiento. Tienen un mayor contenido energético pero son más caros de producir porque requieren más insumos. Estos pueden clasificarse en: cultivos azucareros, cereales y cultivos de semillas oleaginosas. 

Los biorresiduos son combustibles productos de desecho de otras actividades. Suelen tener un menor contenido energético, pero se encuentran en grandes medidas. Ejemplo: residuos agrícolas.

Actualmente el etanol es la fuente de bioenergía  líquida más común y se la mezcla con combustibles para obtener menor emisión de carbono ya que se quema de forma más limpia.

Otra forma de energía que está avanzando a pasos agigantados es el biocarbon ya que tiene importantes beneficios como el aumento de la fertilidad del suelo, la retención de agua y la producción de energía renovable como subproducto de la fabricación de biocarbón. A pesar del alto potencial del biocarbón para secuestrar carbono, se necesita de una gran inversión de capital. 

Entre las posibles materias primas del biocarbón se encuentran: los residuos forestales, los residuos agrícolas y cultivos leñosos de crecimiento rápido.

El biogás es otro combustible que se genera a partir de la descomposición anaeróbica de la materia orgánica. El biogás se refiere principalmente al biometano (también CH4), que suele estar formado por organismos biológicos llamados patógenos que prosperan en entornos anaeróbicos (sin oxígeno). Sin embargo, el biogás puede referirse a otros gases derivados de la biología, como el hidrógeno y la biosíntesis.

\subsection{Geotermia.
}

Debido a que la tierra pierde calor en forma de vapor, es posible utilizarlo para transformarlo en energía eléctrica mediante 3 procesos: vapor seco, vapor flash y ciclo binario. El más eficaz es la central geotérmica de vapor seco, ya que aplican el vapor directamente a una turbina, que acciona un generador que genera electricidad. El tipo de central geotérmica viene dictado por la naturaleza del recurso (ejemplo: geiser).

Las plantas de vapor flash son el tipo más común de planta de generación de energía geotérmica en funcionamiento. Al igual que las plantas de vapor seco, utilizan un fluido que se encuentra bajo tierra a temperaturas superiores a 360 °F y lo bombean a alta presión a un tanque. El tanque de la superficie se mantiene a una presión mucho menor, lo que hace que parte del líquido se vaporice a gran velocidad. A partir de él, el vapor hace girar la turbina conectada a un generador eléctrico para generar electricidad para su uso. 

Las centrales geotérmicas de ciclo binario se diferencian de la tecnología de los sistemas de vapor seco y vapor flash porque el fluido del depósito geotérmico nunca entra en contacto con la turbina o el generador. Las centrales de ciclo binario utilizan un sistema de bucle cerrado. Al igual que las otras dos centrales, las de ciclo binario prácticamente no emiten nada a la atmósfera.

Debido a que se toma el producto de la corteza terrestre, un problema a gestionar son las emisiones de azufre.

\subsection{Energía de las olas.
}

La energía de las olas puede extraerse de las olas superficiales, de las fluctuaciones de presión de la superficie del agua o de la ola completa. Algunas tecnologías se quedan en una posición fija y dejan que las olas las pasen, y otras siguen las olas y se mueven con ellas. Las olas varían con las diferencias de temperatura y presión. 

Para aprovechar la energía de las olas y generar electricidad es necesario utilizar un convertidor de energía de las olas (WEC). Para ello se utilizan 3 tipos de dispositivos: absorbedores puntuales, atenuadores y terminadores. En los absorbedores puntuales, el CME es similar a una boya y se mueve verticalmente, hacia arriba y hacia abajo; no importa de dónde provengan las olas directas. Los absorbedores puntuales se utilizan sobre todo en ubicaciones de olas cercanas a la costa y son dispositivos no direccionales. En los atenuadores, el CME está orientado a lo largo de la dirección de las olas y genera energía cuando la ola pasa por la longitud del sistema; parece estar en la superficie del agua. Los CME terminadores son para olas en tierra. El diseño actúa como un rompeolas, ya que las olas rompen en una rampa para ser recogidas en un depósito, que luego impulsará una turbina de baja altura.

Cabe resaltar que los proyectos de energía de las olas tendrán que tener en cuenta la perturbación o destrucción de la vida marina, la posible amenaza a la navegación por colisión, la interferencia con la pesca comercial y deportiva, o la degradación de la vista del océano por los dispositivos y las líneas de transmisión.

\subsection{Energía mareomotriz.
}

Este tipo de energía aprovecha el flujo de las mareas, que suben y bajan. El principal motor de esta energía son las variaciones de la atracción gravitatoria, principalmente de la luna pero también del sol. La energía cinética se obtiene, ya sea por el flujo que pasa a través de las turbinas o por el embalse de la marea detrás de una estructura similar a una presa. Para ello se utilizan generadores de corriente mareomotriz, que son dispositivos que captan la energía de un flujo o corriente de agua.

Uno de los principales problemas a los que se enfrenta la energía hidroeléctrica ante el cambio climático es el nivel de nieve y la disponibilidad de agua, ya que predicciones prevén que el nivel de nieve aumente en algunas partes del mundo. Esto podría dar lugar a una menor disponibilidad de agua durante las épocas más secas del año, lo que significaría una menor disponibilidad de energía hidroeléctrica.

\section{Conclusión.}
Una vez que ya hemos distinguido los distintos tipos de energía renovables, podemos observar el caso particular de Japón, el cual busca eliminar la energía nuclear para la realización de actividades. Esto se debe a la catástrofe sufrida en el año 2011, en el cual, a pesar que los reactores estaban debidamente construidos para soportar el terremoto, fue el tsunami el elemento no contemplado. 

Sin embargo, nosotros creemos que podrían mantener a la energía nuclear dentro de sus recursos, pero además, podría considerar la implementación de la energía de las olas. Para la incorporación de este tipo de energía se podría utilizar a los generadores con dos finalidades: la generación de energía propiamente dicha y como barrera contra el agua proveniente de un tsunami o avalancha. Esto lo podemos encontrar en otras zonas como en Venecia, donde gracias a esta barrera ha permitido evitar las inundaciones en la ciudad.
Por el contrario, también debemos tener en cuenta que este equipo no puede ser instalado en cualquier país costero, ya que si consideramos el caso de Chile (el cual posee una plataforma continental muy empinada) careceríamos de una base firme para su correcta instalación. Una misma conclusión se podría deducir para los generadores eólicos, los cuales necesitan de un cableado por debajo de la tierra para su funcionamiento. Otro país donde se dificulta la obtención de energía a partir de las olas es Argentina, debido a que su instalación es muy costosa.

Por último, para poder implementar los distintos tipos de energía debemos considerar la vialidad que tenga en la región, es decir, considerar costos de instalación y las cantidades del recurso natural que se nos ofrece. Pero, a pesar de la dificultad de hacer un cambio radical en el pasaje de las energías actuales a las energías limpias, este proceso debe ser acompañado para así poder desprendernos de estas generadoras de contaminación.

\section{Revisión.}
El anterior escrito fue revisado por la alumna Victoria Palma.
\end{document}
